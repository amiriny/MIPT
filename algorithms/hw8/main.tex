\documentclass{article}
\usepackage[T2A]{fontenc}
\usepackage[utf8]{inputenc}   
\usepackage[english, russian]{babel}

% Set page size and margins
% Replace `letterpaper' with`a4paper' for UK/EU standard size
\usepackage[a4paper,top=2cm,bottom=2cm,left=2cm,right=2cm,marginparwidth=1.75cm]{geometry}

\usepackage{amsmath}
\usepackage{graphicx}
\usepackage[colorlinks=true, allcolors=blue]{hyperref}
\usepackage{amsfonts}
\usepackage{amssymb}
% \usepackage[left=1cm,right=1cm,top=1cm,bottom=1cm]{geometry}
\usepackage{hyperref}
\usepackage{seqsplit}
\usepackage[dvipsnames]{xcolor}
\usepackage{enumitem}
\usepackage{algorithm}
\usepackage{algpseudocode}
\usepackage{algorithmicx}
\usepackage{mathalfa}
\usepackage{mathrsfs}
\usepackage{dsfont}
\usepackage{caption,subcaption}
\usepackage{wrapfig}
\usepackage[stable]{footmisc}
\usepackage{indentfirst}
\usepackage{rotating}
\usepackage{pdflscape}

\usepackage{MnSymbol,wasysym}


\begin{document}

\title{\textbf{Основные алгоритмы. Домашняя работа 8 неделя}}


\author{Зайнуллин Амир}
\maketitle

\section*{Задача №1}
По памяти $\Theta \left(\dfrac{\pi}{\rho} \cdot \dfrac{\sqrt{w^2 + h^2}}{d}\right)$

Для времени сначала мы для каждой точки должны отметить $\dfrac{2 \pi}{\rho}$ точек, а потом пройтись по всему массиву чтобы найти максимум. 
По времени $\Theta \left(\dfrac{\pi}{\rho} \cdot \dfrac{\sqrt{w^2 + h^2}}{d} + n \cdot \dfrac{2\pi}{\rho}\right)$
\section*{Задача №2}

Ошибка для одной точки равна 
$$ \varepsilon_i = x_i - a \sin(t_i) $$

Сумма квадратов ошибок
$$ E = \sum_{i = 1}^{n} \varepsilon_i^2 = \sum_{i = 1}^{n}(x_i - a \sin(t_i))^2 $$

Найдем минимум с помощью производной 
$$\frac{dE}{da} = -2 \sum_{i = 1}^{n} (x_i - a \sin(t_i)) \sin(t_i) = 0 $$

$$a\sum_{i=1}^{n}\sin^2(t_i) = \sum_{i=1}^{n}x_i\sin(t_i)$$

$$\hat{a} = \frac{\sum_{i=1}^{n}x_i\sin(t_i)}{\sum_{i=1}^{n}\sin^2(t_i)} $$

\section*{Задача №3} 

Ошибка для одной точки равна 
$$ \varepsilon_i = a \sin(t_i) + b - x_i $$

Сумма квадратов ошибок
$$ E = \sum_{i = 1}^{n} \varepsilon_i^2 = \sum_{i = 1}^{n}(a \sin(t_i) + b - x_i)^2 $$

Найдем минимум с помощью частных производных 
$$\frac{\partial E}{\partial a} = 2 \sum_{i = 1}^{n} (a \sin(t_i) + b - x_i) \sin(t_i) = 0 $$
$$\frac{\partial E}{\partial b} = 2 \sum_{i = 1}^{n} (a \sin(t_i) + b - x_i) = 0 $$

Введем обозначения 
$$ \sum_{i = 1}^{n} \sin(t_i) = S $$
$$ \sum_{i = 1}^{n} \sin^2(t_i) = S^2 $$
$$ \sum_{i = 1}^{n} x_i = X $$
$$ \sum_{i = 1}^{n} x_i \sin(t_i) = SX $$

Тогда получим следующую систему уравнений

$$
\begin{cases}
    a S^2 + b S - SX = 0 \\
    a S + bn - X = 0
\end{cases}
$$

Решая, получим 

$$
\begin{cases}
    \hat{a} = \dfrac{nSX - (S)(X)}{n S^2 - (S) ^ 2} \\
    \hat{b} = \dfrac{S^2 (X) - n SX (S)}{nS^2 - (S) ^ 2}
\end{cases}
$$

\section*{Задача №4} 
$\alpha^n$ - вероятность во все разы выбрать плохие точки, значит вероятность успеха равна
$$ P = 1 - \alpha^m $$
$$ \alpha = \sqrt[m]{1 - p} $$

\section*{Задача №5} 
Для сферы $k = 4$, тогда $\alpha^4$ - вероятность выбрать хорошие точки. Так как в секунду работает $M$ раз, то 
$\alpha ^{4M}$ - искомая вероятность.
\end{document}