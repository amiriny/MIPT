\documentclass{article}
\usepackage[T2A]{fontenc}
\usepackage[utf8]{inputenc}   
\usepackage[english, russian]{babel}

% Set page size and margins
% Replace `letterpaper' with`a4paper' for UK/EU standard size
\usepackage[a4paper,top=2cm,bottom=2cm,left=2cm,right=2cm,marginparwidth=1.75cm]{geometry}

\usepackage{amsmath}
\usepackage{graphicx}
\usepackage[colorlinks=true, allcolors=blue]{hyperref}
\usepackage{amsfonts}
\usepackage{amssymb}
% \usepackage[left=1cm,right=1cm,top=1cm,bottom=1cm]{geometry}
\usepackage{hyperref}
\usepackage{seqsplit}
\usepackage[dvipsnames]{xcolor}
\usepackage{enumitem}
\usepackage{algorithm}
\usepackage{algpseudocode}
\usepackage{algorithmicx}
\usepackage{mathalfa}
\usepackage{mathrsfs}
\usepackage{dsfont}
\usepackage{caption,subcaption}
\usepackage{wrapfig}
\usepackage[stable]{footmisc}
\usepackage{indentfirst}
\usepackage{rotating}
\usepackage{pdflscape}

\usepackage{MnSymbol,wasysym}


\begin{document}

\title{\textbf{Основные алгоритмы. Домашняя работа 3-4 неделя}}
\author{Зайнуллин Амир}
\maketitle

\section*{Задана №2}
$$r_i =  \sum_{j=0}^{m-1} p_{j} t_{i+j}\left(p_{j}-t_{i+j}\right)^{2} = \sum_{j=0}^{m-1}\left(p_{j}^{3} t_{i+j}-2 p_{j}^{2} t_{i+j}^{2}+p_{j} t_{i+j}^{3}\right) $$


Решение: 
$\Rightarrow$

Если образец входит в текст в позиции $i$, тогда слагаемое $p_{j} t_{i+j}\left(p_{j}-t_{i+j}\right)^{2}$ в каждой сумме равно 0.
Если символы равны, то $\left(p_{j}-t_{i+j}\right)$ равно 0.Если один из символов джокер, то слагаемое тоже будет равно 0, потому что или $p_{j} = 0$ или $t_{i + j} = 0$. \\

$\Leftarrow$

Понятно, что каждое слагаемое в сумме положительно, т.к по условию $p_{j}$ и $t_{i+j}$ положительны. $\left(p_{j}-t_{i+j}\right)^{2}$ тоже положительно. 
Тогда чтобы вся сумма равнялась 0, необходимо чтобы каждое слагаемое было равно 0. А это возможно только когда символы совпадают или один из символов джокер. То есть образец входит в текст. 

Пункт 2:

% Нам необходимо для каждого i от 0 до $n - m - 1$ посчитать сумму $ \sum_{j=0}^{m-1}\left(p_{j}^{3} t_{i+j}-2 p_{j}^{2} t_{i+j}^{2}+p_{j} t_{i+j}^{3}\right) $. Чтобы это сделать, рассмотрим перемножение двух полиномов.
% $T(x) = t_{n - 1} x^{n - 1} + ... + t_1 x + t_0$ и $P(x) = p_0 x^{m - 1} + ... + p_{m - 2} x + p_{m - 1}$. Мы можем перемножить их за $O(nlogn)$


$ \deltaA = p dV$

\end{document}